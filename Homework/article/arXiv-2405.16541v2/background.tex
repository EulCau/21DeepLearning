\pg{From kernel estimation to optimal transport (OT)}
Define the \emph{kernel estimator} \smash{$\widehat{k}(\boldsymbol{x}_i, \boldsymbol{x}_j) \coloneqq \phi(\boldsymbol{x}_i)^\top \phi(\boldsymbol{x}_j)$}. 
Recall that $\phi(\cdot)$ is computed using random frequencies $\{\boldsymbol{\omega}_i\}_{i=1}^m$, with the space in which they live, distribution from which they are drawn, and manner in which they are combined dependent on the particular kernel being approximated.
The estimator is unbiased provided each frequency $\boldsymbol{\omega}_i$ obeys some marginal distribution $\eta$. 
Importantly, independence of $\{\boldsymbol{\omega}_i\}_{i=1}^m$ is \emph{not} required: any joint distribution with marginals $\eta$ gives an unbiased estimator. We refer to the set of such joint distributions as \emph{couplings}.  

The coupling between the frequencies determines the estimator variance. 
%It is straightforward to see that 
We want to solve:
%Finding the coupling $\mu^*$ that minimises this variance is equivalent to solving
\begin{equation} \label{eq:first_ot_formulation}
    \textrm{minimise } \mathcal{I}(\mu) =   \mathbb{E}_{\boldsymbol{\omega}_{1:m} \sim \mu}  c(\boldsymbol{\omega}_{1:m}) \quad \textrm{ for } \quad \mu \in \Lambda_m(\eta),
\end{equation}
where we defined the \emph{cost function} $c(\boldsymbol{\omega}_{1:m}) \coloneqq \left( \phi(\boldsymbol{x})^\top \phi(\boldsymbol{y}) \right)^2$ and $\Lambda_m(\eta)$ denotes the set of couplings of $m$ random variables with marginal measures $\eta$. 
This is precisely the \emph{Kantorovich formulation} of a multi-marginal OT problem (see Eq.~4 of the seminal OT text of \citet{villani2021topics}).
We will generally consider cost functions where the minimiser exists and we want to find efficient new MC couplings, so the task is to find the \emph{optimal coupling} $\mu^* = \arg \min_{\mu \in \Lambda_m(\chi_d)} \left [ \mathbb{E}_{\omega_{1:m} \sim \mu}  c(\omega_{1:m}) \right]$ with the smallest estimator variance.
The relationship between variance reduction and OT was also noted by \citet{rowland2018geometrically} in a different context. %the separate context of Gaussian processes.
%estimating the expected value of functions drawn from Gaussian processes.
%with different covariance functions. 

\pg{(Approximately) solving the OT problem} 
The formulation of Eq.~\ref{eq:first_ot_formulation} depends on the particular RF mechanism and kernel being approximated. 
We will show that one can solve it exactly for RFFs and RLFs (\textbf{Sec.~\ref{sec:rffs_and_rlfs}}) and approximately for GRFs (\textbf{Sec.~\ref{sec:grfs}}), which have input domains $\mathcal{X} = \mathbb{R}^d$ ($d$-dimensional Euclidean space) and $\mathcal{X} = \mathcal{N}$ (the set of graph nodes) respectively.
This gives new couplings with lower RF variance than previous algorithms. 


%For many popular RF constructions, the OT problem in Eq.~\ref{eq:first_ot_formulation} is intractable. 
%The cost function and corresponding optimal coupling will in general depend on the choice of RF map and data distribution.
%Moreover, the cost function $c(\boldsymbol{\omega}_{1:m})$ and the set of possible couplings $\Lambda_m(\eta)$ depends on the particular kernel function and RFs being considered; the best coupling wo. 
%A further practical consideration is that, to be useful, $\mu^*$ must be easy to sample from; even if it guarantees lower variance, an expensive sampling mechanism will obviate any efficiency gains from a coupling.
%However, we will see that for certain RFs under particular assumptions we can solve the OT problem \emph{exactly}, obtaining computationally lightweight couplings that provide the smallest possible kernel estimator variance.
%Even if the OT problem is intractable, we can often leverage numerical methods to \emph{approximately} solve Eq.~\ref{eq:first_ot_formulation}. 
%The formulation of these methods will depend on the input domain of the RF being considered.

%\pg{Remainder of the manuscript} We will show how perspectives from OT can improve the convergence of RFFs and RLFs (Sec.~\ref{sec:rffs_and_rlfs}) and GRFs (Sec.~\ref{sec:grfs}), which have input domains $\mathcal{X} = \mathbb{R}^d$ ($d$-dimensional Euclidean space) and $\mathcal{X} = \mathcal{N}$ (the set of graph nodes) respectively.

