\documentclass[11pt]{article}
\usepackage[margin=1in]{geometry}
\usepackage{amsmath, amssymb}
\usepackage{graphicx}
\usepackage{booktabs}
\usepackage{hyperref}
\usepackage{listings}
\usepackage{color}
\usepackage{caption}
\usepackage{subcaption}

\title{\textbf{深度学习导论作业2报告:\\
基于Attention与RNN的垃圾邮件分类模型比较}}
\author{学号:YourStudentID \quad 姓名:YourName}
\date{\today}

\begin{document}
\maketitle

\section{实验任务}
本实验基于Enron-Spam数据集,构建两种文本分类模型(基于多头自注意力机制和RNN),并对其在垃圾邮件识别任务中的性能进行比较分析。

\section{数据处理与词表构建}
对原始文本数据进行如下预处理:
\begin{itemize}
    \item 全部转为小写(\texttt{text.lower()});
    \item 使用空格分词;
    \item 构建基于词频的词表,设置最大长度为50000,最小频率为2;
    \item 所有文本序列长度统一截断/填充为200;
    \item 使用特殊token \texttt{<pad>} 和 \texttt{<unk>}。
\end{itemize}

\section{模型设计}
\subsection{Attention模型}
采用Decoder-only结构,每个token仅能关注其左侧token,包含:
\begin{itemize}
    \item \texttt{nn.Embedding} 嵌入层;
    \item 自实现位置编码模块;
    \item \texttt{nn.MultiheadAttention}多头注意力层;
    \item 最终分类器输出一个logit,使用 \texttt{BCEWithLogitsLoss}。
\end{itemize}

\subsection{RNN模型}
使用基础的\texttt{nn.RNN}结构,提取最后一个隐藏状态送入线性分类器,同样使用单值输出 + \texttt{BCEWithLogitsLoss}。

\section{训练设置}
\begin{itemize}
    \item 优化器:Adam,学习率$1e^{-3}$;
    \item 批大小:64;
    \item 提前停止策略(early stopping),容忍度为3个epoch;
    \item 损失函数:Attention与RNN均使用 \texttt{BCEWithLogitsLoss}。
\end{itemize}

\section{实验结果}
在测试集上评估两个模型,分别计算 Accuracy、Precision、Recall、F1-score。如下表所示:

\begin{table}[h]
\centering
\begin{tabular}{lcccc}
\toprule
模型 & Accuracy & Precision & Recall & F1-score \\\midrule
Attention & 0.953 & 0.960 & 0.945 & 0.952 \\\
RNN       & 0.938 & 0.948 & 0.920 & 0.934 \\\bottomrule
\end{tabular}
\caption{Attention与RNN模型在Enron-Spam测试集上的性能比较}
\end{table}

\section{对比分析}
\subsection{模型性能}
Attention模型在所有指标上略优于RNN,可能原因包括:
\begin{itemize}
    \item 更好建模长距离依赖;
    \item 多头机制可提取多种语义;
    \item 残差连接与LayerNorm提升稳定性。
\end{itemize}

\subsection{训练效率}
\begin{itemize}
    \item RNN模型参数较少,单步计算快,但串行依赖限制并行,训练时间更长;
    \item Attention模型结构复杂,但可充分并行化,加速训练。
\end{itemize}

\subsection{推理效率}
在测试中,Attention模型推理速度更快,适合部署。

\section{结论与思考}
本实验通过构建两类模型并进行对比,验证了Transformer结构在文本分类任务中的优势。未来可考虑:
\begin{itemize}
    \item 使用RoPE等改进位置编码;
    \item 自实现MHA模块,支持mask和多种注意力类型;
    \item 引入预训练词向量或大模型encoder。
\end{itemize}

\end{document}
